\documentclass{article}

\usepackage[ngerman]{babel}
\usepackage[utf8]{inputenc}
\usepackage[T1]{fontenc}
\usepackage{hyperref}
\usepackage{csquotes}

\usepackage[
    backend=biber,
    style=apa,
    sortlocale=de_DE,
    natbib=true,
    url=false,
    doi=false,
    sortcites=true,
    sorting=nyt,
    isbn=false,
    hyperref=true,
    backref=false,
    giveninits=false,
    eprint=false]{biblatex}
\addbibresource{../references/bibliography.bib}

\title{Notizen zum Projekt Data Ethics}
\author{Yanic Nef}
\date{\today}

\begin{document}
\maketitle

\abstract{
    Dieses Dokument ist eine Sammlung von Notizen zu dem Projekt. Die Struktur innerhalb des
    Projektes ist gleich ausgelegt wie in der Hauptarbeit, somit kann hier einfach geschrieben
    werden, und die Teile die man verwenden möchte, kann man direkt in die Hauptdatei ziehen.
}

\tableofcontents

\input{section_ai.tex}

\section{Was ist eine KI?}
\begin{itemize}
    \item Was kann eine KI?
    \item Wer kann KI nutzen?
\end{itemize}

\section{Wie funktioniert KI?}
\begin{itemize}
    \item Woher nimmt KI ihre Informationen?
    \item Wie verwertet KI ihre Informationen?
\end{itemize}

\section{Wie kann KI helfen?}
\begin{itemize}
\item Gesundheitswesen
\item Bildung
\item Forschung
\item Nachhaltigkeit
\item Kommunikation
\end{itemize}

\section{Wie kann KI ethisch schaden?}
\begin{itemize}
    \item Krieg
    \item Wirtschaftlich
    \item Datenschutz
    \item Missinformationen und Deepfakes
\end{itemize}

\section{Schlussfolgerung}
Kommt auf Gebrauch an

\printbibliography

\end{document}

\documentclass{report}

\usepackage[ngerman]{babel}
\usepackage[utf8]{inputenc}
\usepackage[T1]{fontenc}
\usepackage{hyperref}
\usepackage{csquotes}
\usepackage[a4paper]{geometry}

\usepackage[
    backend=biber,
    style=apa,
    sortlocale=de_DE,
    natbib=true,
    url=false,
    doi=false,
    sortcites=true,
    sorting=nyt,
    isbn=false,
    hyperref=true,
    backref=false,
    giveninits=false,
    eprint=false]{biblatex}
\addbibresource{../references/bibliography.bib}


\title{KI im Ethischen Gebrauch}
\author{Yanic Nef}
\date{\today}


\begin{document}

\maketitle

\abstract{
    In diesem Dokument erkläre ich grob wie KI funktioniert und wie sie sich Positiv, so wie negativ auf verschiedene Ethische Aspekte auswirkt.
}

\tableofcontents

\chapter{Einleitung}

In den letzten Jahren fing KI an sich sehr stark zu verbreiten und gilt für viele als ein wichtiger Aspekt im Alltag. Aber wie funktioniert sie und kann sie auch gefährlich sein? In dieser Arbeit erkläre ich ihnen, was KI ist, wie sie funktioniert und wie sie sich Positiv so wie negativ auf Menschen und die Welt auswirken kann.


\chapter{Allgemein}
\label{chap:allgemein}

In diesem Kapitel werde ich grob erklären wie KI funktioniert und woher sie die Daten hat.

\section{Was ist eine KI?}

Eine KI (künstliche Intelligenz) oder auch AI (Artificial Intelligence) genannt, ist eine Technologie, welche die Menschliche Intelligenz versucht zu simulieren und für sie Fragen beantworten kann, welche aus verschiedensten Quellen zusammengetragen wurden. solch logisches denken, konnten bisher nur Menschen zustandebringen. KI ist für jeden zugänglich, jedoch sind gewisse Dienstleistungen kostenpflichtig.

\section{Wie funktioniert KI?}

KI Probiert mit verschiedensten Technologien Informationen zusammenzutragen. Diese Technologien sind komplex und mir ist es nicht möglich sie alle genaustens zu erklären, aber ich werde versuchen grob darüber zu berichten.

\subsection{Neuronale Netzwerke}

Wie bei dem Gehirn, wird bei KI ein Neuronales Netzwerk verwendet. Dazu werden künstliche neuronale Netzwerke genutzt. Das heisst künstliche Neuronen ahmen Gedankengänge mithilfe von Algorithmen nach. Diese Neuronen bilden mehrere Schichten, die Informationen verarbeiten. Das ganze ist noch viel komplexer aber das soll ja nicht meinen ganzen Aufsatz aufbrauchen.

\chapter{KI im Ethischen Gebrauch}

\section{Wie kann KI Ethisch helfen?}

KI kann auf verschiedene Wege Ethisch Hilfe leisten, welche die Welt und die Menschen bereichert. Ich berichte über verschiedene Aspekte aber natürlich gibt es noch vieles mehr bei dem KI helfen kann.

\subsection{Gesundheitswesen}

KI kann dabei helfen Krankheiten oder andere Probleme mit der Gesundheit zu diagnostizieren. Mittlerweile ist KI sogar dazu fähig Röntgenbilder oder MRTs auszuwerten. Jedoch sollte dies immer durch einen Arzt nachgeprüft werden, da die KI nicht immer 100 prozentig richtig liegt. 

\subsection{Bildung}

Auch in der Bildung, kann KI helfen. Informationen, welche beispielsweise für eine Prüfung benötigt werden, kann KI einfach zusammenfassen. Es gibt auch KIs, welche massgeschneiderte Lernmethoden für sie anfertigen kann und wie ein Lehrer funktioniert. 

\subsection{Forschung}

Durch den grossen Zugang zu Informationen, kann KI wichtige Daten zusammenfassen um Experimente zu planen und Hypothesen zu generieren. Ausserdem ist es durch diese grosse Menge an Informationen einfacher auf den aktuellen Stand der Forschung zu kommen, falls eine Person etwas aufholen muss. 

\subsection{Nachhaltigkeit}

Durch Analyse des Verbrauchs verschiedener Organismen kann KI beispielsweise den Gebrauch von Wasser und Dünger optimieren um weniger zu verschwenden. Ausserdem ist es KI möglich den Energieverbrauch von Gebäuden zu reduzieren.

\subsection{Kommunikation}

Aufgrund der Fähigkeiten von KI, verschiedene Sprachen zu verstehen, ist es durch KI möglich die Kommunikation von Menschen, welche durch ihre Sprache nicht kommunizieren können, zu ermöglichen. Ausserdem kann KI helfen behinderte zu inkludieren, indem sie beispielsweise in einer Prothese eingesetzt wird, welche realistische Bewegungen des menschlichen Körpers nachahmen kann. Durch Technologien wie Sprach zu Text, können auch Personen, welche andere Einschränkungen haben inkludiert werden und normal kommunizieren.

\section{Wie kann KI schaden?}

Zwar kann KI helfen, aber darurch dass KI von jedem gebraucht werden kann, auch drastisch schaden. folgend sind einige Beispiele dazu.

\subsection{Krieg}

KI ist dazu fähig, Peronen zu erkennen und beispielsweise durch Drohnen sie aufzusuchen und zu ermorden. Das kann zwar helfen, Kriegsverbrecher zu finden, jedoch auch um unschuldige Menschen in Kriegsgebieten zu ermorden. 

\subsection{Wirtschaftlich}

Durch viele automatisierte Aufgaben, kann KI bereits Menschen ersetzen. Durch dieses Ersetzen verlieren viele Menschen ihren Arbeitsplatz und sie werden nicht mehr gebraucht. Die Firmen, welche bereits über viel Geld besitzen, können sich automatisierte KI leisten, während kleinere Firmen mit limitierterem Budget also zurückbleiben.

\subsection{Sicherheit}



\subsection{Missinformationen und Deepfakes}

\printbibliography

\end{document}

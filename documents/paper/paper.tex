\documentclass{report}

\usepackage[ngerman]{babel}
\usepackage[utf8]{inputenc}
\usepackage[T1]{fontenc}
\usepackage{hyperref}
\usepackage{csquotes}
\usepackage[a4paper]{geometry}

\usepackage[
    backend=biber,
    style=apa,
    sortlocale=de_DE,
    natbib=true,
    url=false,
    doi=false,
    sortcites=true,
    sorting=nyt,
    isbn=false,
    hyperref=true,
    backref=false,
    giveninits=false,
    eprint=false]{biblatex}
\addbibresource{../references/bibliography.bib}


\title{Ethik im Umgang mit Daten}
\author{Yanic Nef}
\date{\today}


\begin{document}

\maketitle

\abstract{
    In diesem Dokument erkläre ich, wie KI funktioniert und wie sie sich Positiv, so wie negativ auf verschiedene Ethische Aspekte auswirkt.
}

\tableofcontents

\chapter{Einleitung}

Hier kommt die Einführung. Der Text hier sollte eigentlich noch viel länger sein, so das hier nicht so merkwürdige Umbrüche entstehen.

Ich kann weitere Kapitel auch importieren.

\chapter{Allgemein}
\label{chap:allgemein}

In diesem Kapitel werde ich grob erklären wie KI funktioniert und woher sie die Daten hat.

\section{Was ist eine KI?}

Eine KI (künstliche Intelligenz) oder auch AI (Artificial Intelligence) genannt, ist eine Technologie, welche die Menschliche Intelligenz versucht zu simulieren und für sie Fragen beantworten kann, welche aus verschiedensten Quellen zusammengetragen wurden. solch logisches denken, konnten bisher nur Menschen zustandebringen.

\section{Wie funktioniert KI?}

KI Probiert mit verschiedensten Technologien Informationen zusammenzutragen.


\printbibliography

\end{document}

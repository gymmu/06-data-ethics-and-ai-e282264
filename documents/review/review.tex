\documentclass{article}

\usepackage[ngerman]{babel}
\usepackage[utf8]{inputenc}
\usepackage[T1]{fontenc}
\usepackage{hyperref}
\usepackage{csquotes}

\usepackage[
    backend=biber,
    style=apa,
    sortlocale=de_DE,
    natbib=true,
    url=false,
    doi=false,
    sortcites=true,
    sorting=nyt,
    isbn=false,
    hyperref=true,
    backref=false,
    giveninits=false,
    eprint=false]{biblatex}
\addbibresource{../references/bibliography.bib}

\title{Review des Papers "Ethik und Daten der KI" von \dots}
\author{Luka Adjancic}
\date{\today}

\begin{document}
\maketitle

\section{Was wurde gut gemacht?}
    KI wird knapp aber genügend erklärt und anhand von Zitaten unterstützt. Mir gefällt, wie Bilder miteingebracht wurden und die Erklärungen unterstützen. Das Trolley Problem, welches ein Problem ist, bei welchem es keine "richtige" Lösung gibt, sehe ich als sehr passendes Beispiel an, da es ebenfalls bei Menschen schon für Diskussionen sorgte. Die Fähigkeiten und Grenzen von KI werden im kompletten Text sehr gut erklärt und anhand von Beispielen belegt.

\section{Was könnte noch verbessert werden?}

Teilweise finde ich die Zitate unnötig und sie passen nicht immer zu dem was gesagt wurde. An wenigen Stellen sind Rechtschreibfehler und Grammatische Fehler vorhanden, jedoch liegen die Informationen im Vordergrund.

\section{Schlussfolgerung}
Alles in allem wurde das Thema sehr gut erfasst und erklärt. Alles wurde genügend gut erklärt, so dass man mithalten kann. 

\printbibliography

\end{document}
